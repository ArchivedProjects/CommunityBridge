\documentclass[letterpaper,12pt]{article}
\usepackage{amsfonts,amsmath,amssymb,fancyhdr,datetime}

\usepackage[pdftex]{graphicx} % Great for including graphics
\usepackage{wrapfig}          % Wrapper for figures

\usepackage{upquote}
\usepackage{verbatim}

% Relies on the datetime package to reformat the date the way I like.
\newdateformat{iaindateformat}
              {\THEYEAR\hspace{.5em}\monthname[\THEMONTH]\hspace{.5em}\THEDAY}
\iaindateformat

\oddsidemargin 0.0in
\evensidemargin 0.0in
\textwidth 6.5in

\topmargin 0.0in
\headheight 1.0in
\textheight 9.0in

\renewcommand{\choose}[2]{\genfrac{(}{)}{0pt}{}{#1}{#2}}

\renewcommand{\labelitemii}{$\circ$}
\renewcommand{\thesection}{\arabic{section}}

\pagestyle{fancy}
  \setlength\headheight{14.5pt}
  \renewcommand{\headrulewidth}{0.4pt}
  \renewcommand{\footrulewidth}{0.4pt}
  \fancyhead[R]{\bf Developer \thepage}
  \fancyhead[L]{\bf }
  \fancyfoot[R]{\bf Generated: \today\hspace{.5em}\currenttime}
  \fancyfoot[L]{\bf Community Bridge}
  \fancyfoot[C]{\bf \thepage}
\setlength{\parindent}{0pt} 
\setlength{\parskip}{0.5em} 
\begin{document}
  \section{Overview}
  Throughout this document, the term ``web application'' is used. The web
  application can be any forum, CMS, blogging software, etc. that uses MySQL for
  its database.
  
  These are the major features of Community Bridge:
  \begin{enumerate}
    \item Minecraft player to web application user ``linking''.
    \item Group synchronization (currently a misnomer, since the
          ``synchronization'' is only one way)
    \item Recording statistical information about the players in web
          application custom fields.
    \item Kicking of players who are banned from the web application.
  \end{enumerate}
  
  The first of the features is essential, without it, none of the other
  features of the plugin would work. Given that the linking feature's
  importance, it is the first thing to evaluate when ensuring that
  CommunityBridge is operating correctly.

  \clearpage
  \subsection{Proposed config.yml Preamble}
  \verbatiminput{config-preamble.txt}

  \clearpage
  \subsection{Proposed config.yml Database Section}
  \verbatiminput{config-database.txt}

  \clearpage
  \section{Linking Feature}
  For each of the following configurations, CB should:
  \begin{itemize}
    \item Recognize a player as not being registered, e.g., not present in the
    web application's database.
    \item Correctly identify a player as being registered--identifying the
    correct user.
    \item {\bf (NYI)} If authentication is off {\bf DO NOT} force authentication.
    \item {\bf (NYI)} If authentication is on, handle a valid authentication by...
    \item {\bf (NYI)} If authentication is on, respond to an failed authentication attempt by...
  \end{itemize}

  Possible Configurations (order of complexity):
  \begin{itemize}
    \item Minecraft playername and web application username are required to be
      the same. Required information: users table, user id column, username
      column. Configuration Name: same\_name.
    \item Minecraft playername and web application username can be different;
      the Minecraft playername is in a separate column on the users table.
      Required information: users table, user id column, minecraft playername
      column. Configuration Name: diff\_name\_same\_table.
    \item Minecraft playername and web application username can be different;
      the Minecraft playername is on a different table in its own column
      (frequently, this is because it is implemented through a custom profile
      field in the web application). Required information: tablename, user id
      column, minecraft playername column.
      Configuration Name: diff\_name\_diff\_table
    \item Minecraft playername and web application username can be different;
      the Minecraft playername is on a different table that data is stored in
      the key-value format. Required information: tablename, user id column,
      key column, value column, data key name.
      Configuration Name: diff\_name\_diff\_table\_keyed
  \end{itemize}
  \clearpage
  
  \subsection{Proposed config.yml Section}
  \verbatiminput{config-linking-feature.txt}
  
  \subsection{Proposed messages.yml section}
  Messages related to the linking feature in the messages.yml.
  \verbatiminput{message-linking-feature.txt}

  \clearpage
  \section{Group Synchronization}
Many web systems have a notion of a ``primary'' group. However, very few of the
permissions plugins do. This is not a problem when the synchronization is
``Forums override Minecraft'', but it presents a difficulty the other direction,
``Minecraft override Forums''.

	\clearpage
  \subsection{Proposed config.yml Section}
  \verbatiminput{config-group-synchronization.txt}
  
  \clearpage
  \subsection{Proposed messages.yml Section}
  \verbatiminput{message-group-synchronization.txt}

  \clearpage
  \appendix
  \section{Full Proposed config.yml}
  \verbatiminput{config-preamble.txt}
  \verbatiminput{config-database.txt}
  \verbatiminput{config-linking-feature.txt}
  \verbatiminput{config-group-synchronization.txt}

  \clearpage
  \section{Full Proposed messages.yml}
  \verbatiminput{message-linking-feature.txt}
  \verbatiminput{message-group-synchronization.txt}
\end{document}
